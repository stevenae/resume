% LaTeX file for resume 
% This file uses the resume document class (res.cls)

\documentclass[12pt]{res} 
\usepackage[usenames, dvipsnames]{color}


%\usepackage{helvetica} % uses helvetica postscript font (download helvetica.sty)
\usepackage{newcent}   % uses new century schoolbook postscript font 
\newsectionwidth{8pt}  % So the text is not indented under section headings
\setlength{\textheight}{10.2in} % set text height big enough for box
\topmargin=-1mm       % to start box .5in from top of page
\oddsidemargin=-8mm   % to start box .5in from left of page
%\renewcommand{\baselinestretch}{1.05}
\begin{document}

%%%%%%%%%%%%%%%%%%%%%%%%%%%%%%%%%%%%%%%%%%%%%%%%%%%%%%%%%%%%%%%%%%%%%%%%%%%%
% The following lines define \boxaround, used to draw a box on the page.
% The parameter is the entire text of the resume. Must fit on one page!
%
% \boxaroundhmargin is the left & right margin around the text inside the box.
% \boxaroundvmargin is the top & bottom margin around the text inside the box.
% \boxrulethickness controls thickness of line used to draw the box.
% You can change these 3 things in the lines below:
%%%%%%%%%%%%%%%%%%%%%%%%%%%%%%%%%%%%%%%%%%%%%%%%%%%%%%%%%%%%%%%%%%%%%%%%%%%%%
\newdimen\boxrulethickness\newdimen\boxaroundhmargin\newdimen\boxaroundvmargin
\boxrulethickness=.2pt        %controls thickness of line 
\boxaroundhmargin=9pt        % about a half inch
\boxaroundvmargin=25pt        % to fit more text on page, make this smaller
%%%%%%%%%%%%%%%%%%%%%%%%% Don't read this stuff %%%%%%%%%%%%%%%%%%%%%%%%%%%%%%
\hsize=6.86in \vsize=12in             % use bigger dimensions for box
\newbox\MACboxA  \newdimen\MACdimenA
% \borderandboxit is used inside \boxaround:
\def\borderandboxit#1#2#3{\vbox{\hrule height#2\hbox{\vrule width#2\hskip#1\hskip-#2%
  \vbox{\vskip#1\relax#3\vskip#1}\hskip#1\hskip-#2\vrule width#2}\hrule height#2}}
%
\long\def\boxaround#1{\vskip6pt
  {\MACdimenA=\hsize \advance\MACdimenA by-\boxaroundhmargin
   \advance\MACdimenA by-\boxaroundhmargin   % once for each side
   \setbox\MACboxA=\hbox to \hsize{\hskip\boxaroundhmargin%\hss
                     \vbox{\hsize=\MACdimenA
                           \vskip\boxaroundvmargin #1
                           \vskip\boxaroundvmargin}\hss}%
   \borderandboxit{0pt}\boxrulethickness{\box\MACboxA}}%
  \vskip2pt plus0pt minus0pt
}
%%%%%%%%%%%%%%%%%%%  End of \boxaround macro %%%%%%%%%%%%%%%%%%%%%%%%%%%%%%%%%
\boxaround{ % put the text on the page inside a box  

\begin{resume}
\begin{list}{}{\leftmargin=1em}
\item \centering{ Steven Ellis}
\item
\item
\item{\dotfill Experience\dotfill
\begin{list}{-}{\leftmargin=1em}
\item I was a data scientist for the United States Digital Service (January 2023\,- June 2023). I supported the White House Working Group on Puerto Rico by analyzing spending data.

\item I was the second employee at GoBuild (July  2020\,-\,June 2021), a seed-stage startup. I developed and piloted the startup's product, a tool which used computer vision to estimate the effect of home renovations on resale value. We successfully raised a seed round based on this prototype.

\item I was the head of data science at Divvy Homes (September  2018\,-\,June 2020). I built and deployed online data services to determine home pricing and financing approval. These tools contributed to a successful series B fundraise. I also hired and managed two data scientists.

\item I was the sixth data scientist at Uber (May  2014\,-\,December  2016).  I quantified the value of business partnerships via causal inference methods and resolved hazardous flaws in Uber's feature rollout infrastructure.

\item I was a quantitative user experience researcher at Google (April  2011\,-\,April  2014).  While there, I guided pricing and product strategy via discrete-choice methods. I also identified surrogate geographic research regions via exact-test statistics.
\end{list}}
\item
\item{\dotfill Competencies\dotfill 
\begin{list}{-}{\leftmargin=1em}
\item Online machine learning service development and deployment. 
\item Offline business process creation and iteration. 
\item Parametric and non-parametric modeling. 
\item Descriptive and predictive statistics. 
\item Team-building.

\end{list}}
\item
\item{\dotfill Education\dotfill 
\begin{list}{-}{\leftmargin=1em}
\item Brown University 2011, Applied Psychology, Magna Cum Laude. 
\end{list}}
\item
\item
\end{list}

\end{resume}
} %    end the material being boxed.
\end{document}


